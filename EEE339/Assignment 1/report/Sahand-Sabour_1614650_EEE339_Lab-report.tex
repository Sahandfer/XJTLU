 \documentclass[12pt,a4paper]{article}
\usepackage{amsmath}
\usepackage{amssymb}
\usepackage{epstopdf}
\usepackage{inputenc}
\usepackage{graphicx}
\usepackage{titletoc} 
\usepackage{fancyhdr}   
\usepackage[a4paper,pdftex]{geometry}	
\usepackage[english]{babel}
\usepackage{xcolor} 
\usepackage{enumerate}
\usepackage{fix-cm} 
\usepackage[notlof]{tocbibind}
\usepackage{amsmath}
\usepackage{listings}
\usepackage{float}
\usepackage{enumitem}
\usepackage{xcolor}
\usepackage{listings}
\definecolor{vgreen}{RGB}{104,180,104}
\definecolor{vblue}{RGB}{49,49,255}
\definecolor{vorange}{RGB}{255,143,102}
\renewcommand\lstlistingname{Appendix}
\renewcommand\lstlistlistingname{Appendix}

\lstdefinestyle{verilog-style}
{
	language=Verilog,
	basicstyle=\small\ttfamily,
	keywordstyle=\color{vblue},
	identifierstyle=\color{black},
	commentstyle=\color{vgreen},
	moredelim=*[s][\colorIndex]{[}{]},
	frame=single,
	literate=*{:}{:}1
}

\makeatletter
\newcommand*\@lbracket{[}
\newcommand*\@rbracket{]}
\newcommand*\@colon{:}
\newcommand*\colorIndex{%
	\edef\@temp{\the\lst@token}%
	\ifx\@temp\@lbracket \color{black}%
	\else\ifx\@temp\@rbracket \color{black}%
	\else\ifx\@temp\@colon \color{black}%
	\else \color{vorange}%
	\fi\fi\fi
}
\makeatother

\usepackage{trace}

\usepackage{subcaption}
\begin{document}
	\begin{titlepage}
		\begin{center}
			\includegraphics[scale=1.5]{figures/CoverSheet}\\
			\bf{ \small{DEPARTMENT OF ELECTRICAL AND ELECTRONIC ENGINEERING} }
		\end{center}
		
		\vspace{4cm}
		\centering
		\textbf{\Huge Assignment 1}
		
		\vspace{1cm}
		
		{\Large Digital Clock}
		
		\vspace{4cm}
		
		\textbf{\LARGE Sahand Sabour}
		
		\vspace{2cm}
		
		\textbf{\large EEE339}
		
		\vspace{0.5cm}
		
		{\large Digital System Design with HDL(I)}
		
		\vspace{1.5cm}
		
		\textbf{\large Student ID}
		
		\vspace{0.5cm}
		
		{\large 1614650}
		
		\vfill
		
	\end{titlepage}
	
	\noindent \textbf{\Large Introduction}
	\vspace{0.2cm}
	
	\noindent Programmable Logic Devices (PLDs) have obtained an essential role in modern digital systems industry, mainly due to their ability to implement both combinational and sequential logic circuits. Evidently, a Hardware Description Language (HDL) would be used to describe such hardware and accordingly, program PLDs to demonstrate the desired functions. In this project, Verilog was used due to its simplicity and similarity to the C language. In addition, the apparatus for this assignment consisted of a DE1 ALTERA board, its power and USB cable as well as the Quartus II software.
	In this assignment, each group of two students were provided with a DE1 board and were tasked to create a digital clock with the following main functions:
	\begin{enumerate}
		\item 4-digit minute and second display.
		\item Time setting.
		\item A stopwatch with 4-digit second and tenth of a second display.
	\end{enumerate}
	In addition, the students were tasked to implement additional functions to improve the digital clock design. Hence, the following functions were determined to be implemented:
	\begin{enumerate}
		\item Up/Down counting.
		\item Timer with flashing LED lights.
		\item Simultaneous operation of both clock and stopwatch with changeable modes: allow the user to choose what is displayed.
	\end{enumerate}

	\noindent Hence, the following hardware overview diagram was designed (Figure 1).
	
	\begin{figure}[H]
		\centering
		\includegraphics[height=4cm,width=10cm]{figures/overview}
		\caption{Hardware overview diagram for Digital Clock system}
	\end{figure}

	\noindent It is believed that this assignment was intended to assess each student's understanding of the subject, challenge the ability to design functional hardware, and provide a valuable learning opportunity to explore Verilog as a HDL.
	
	\vspace{1cm}
	
	\noindent \textbf{\Large Methodology}
	\vspace{0.2cm}
	
	\noindent In this section, a thorough explanation of each block of the design, including its inputs, outputs, and operating logic, is provided. As the Altera board is equipped with four 7-segment displays, $Q_{0-3}$ would be used to reference each display, with $Q_{0}$ being the right-most display and $Q_{3}$ being the left-most display. Hence, it is suggested that these notations be noted for future references. In addition, the connection between the components as well as the pin assignments for the whole circuit is provided at the end of this section. 
	
	\vspace{0.5cm}
	
	\noindent \textbf{\large Second and Decisecond Counter Design}
	\vspace{0.2cm}
	
	\noindent The Altera board is equipped with two oscillators which produce 27MHz and 50MHz clock signals respectively. The prior oscillator was implemented in this assignment. Therefore, each 27000000 clock cycles would account for one second. Accordingly, two counters were designed to output a HIGH signal for each 2700000 and 27000000 clock signals respectively. Consequently, this would allow the other components to be enabled every tenth of a second (Decisecond) or every second depending on their intended functions (check Figure 2, Appendix 1\&2).
	
	\begin{figure}[H]
		\centering
		\includegraphics[height=4cm,width=8cm]{figures/counters}
		\caption{Second and Decisecond Counters diagram}
	\end{figure}

	\vspace{-0.3cm}	
	\noindent \textbf{\large Clock Design}
	\vspace{0.2cm}
	
	\noindent \textbf{Inputs}
	\vspace{0.1cm}
	
	\noindent Initially, in order to design a functional clock (MM:SS display), the following inputs were required:
	
	\vspace{-0.2cm}
	\begin{enumerate}
		\item \textbf{Enc:} allows the clock to operate every second.
		\item \textbf{Clk:} the 27MHZ clock signal produced by the oscillator.
		\item \textbf{Ent:} allows the clock to operate when set to HIGH, pauses the clock operation when set to LOW. 
	\end{enumerate}

\vspace{-0.2cm}
	\noindent The prior two inputs can be obtained from the Altera board itself as clock signal is produced by the 27MHz oscillator and ENC is set to high every 27000000 clock cycles. However, the last input should be determined by flipping a switch, which would allow to the user to resume or pause the operation. Moreover, in order to implement additional functions to the design, the following inputs were added, which were determined by the switches on the Altera board.
	
	\begin{enumerate}
		\item \textbf{Clr:} Restores the clock to the default state (00:00) when set to HIGH.
		\item \textbf{UpDown:} changes the direction of the count; the clock counts upwards when set to LOW or downwards when set to HIGH.
	\end{enumerate}
	
	\noindent With the above inputs, a functional clock with functions of clearing the registers, counting both up and down that operates every second was constructed. In order to add the function of time setting, several measures were to be considered. 
	
	\noindent First, the clock should receive a 4 bit input to determine which digit is to be changed, with each bit representing the digit to be changed when set to HIGH. Second, as declared in the design objectives, concurrent operation between the clock and the stopwatch is to be implemented. Hence, the clock should receive an input corresponding to whether it is being displayed or not. This is necessary as time setting should only be available when the clock is being displayed. Third, an extra counter would be added in order to enhance the user experience. This counter would output a HIGH signal every 0.2 seconds and accordingly, the digits would be increased every 0.2 seconds when a key is hold down. This implementation would allow the user to hold the corresponding keys to set the time rather than pressing them several times. Conclusively, the following inputs were declared:
	\begin{enumerate}
		\item \textbf{timeSetter:} a 4 bit input to declare which digit is being set.
		\item \textbf{mode:} shows that the clock is being displayed when set to LOW.
		\item \textbf{pause:} represents the pause between each increment in the digit being modified and is set to HIGH every 0.2 seconds.
	\end{enumerate}

	\vspace{0.2cm}
	\noindent \textbf{Outputs}
	\vspace{0.1cm}
	
	\noindent As for the outputs, the clock needed to produce two signals: First, a 16 bit output Q, which represents 4 groups of 4 bits, each group corresponding to a digit that is to be displayed, with the left-most 4 bits correspond to the leftmost digit of the clock etc. Second, an output Rco, which declares that the clock has reached the default state in the down counting mode. With the addition of the latter output, the clock could also function as a timer and trigger a following component when it reaches its lower limit (00:00).
	
	\vspace{0.2cm}
	\noindent \textbf{Logic}
	\vspace{0.1cm}
	
	\noindent Initially, the 16 bit output is set to zero. On the positive edge of the clock cycle, the component should analyze the following set of conditional statements and update the outputs accordingly.
	
	\noindent If the clear input (Clr) is set to HIGH, the output will be set to zero. 
	
	\noindent If Clr is set to LOW, the enabled switch is flipped, a second is reached, and the clock is counting upwards, then $Q_{0}$ would be incremented by 1. If $Q_{0}$ reaches its limit, then $Q_{1}$ would be incremented by 1 and Q0 would be reset to 0 etc. As the format of the clock dedicates two digits to display the minutes and two digits to display the seconds, the limits for $Q_{0}$ and $Q_{2}$ would be 9 while the limits for $Q_{1}$ and $Q_{3}$ would be 5. Regarding the down counting operation of the clock, decrements of 1 would be used instead of increments with the limit for all digits being 0. Moreover, each digit would be reset to their respective limits in the up counting mode when they reach 0 in the down counting mode. If all the digits are set to 0 in the down counting mode, Rco would be set to HIGH.
	
	\noindent If none of the above conditions are met and one of the keys on the Altera board is pressed by the user, an increment to the corresponding digit is to be applied. Evidently, the increments follow the same limit as the up counting logic and would set the digit to 0 when the limit is reached.
	  

	\vspace{0.2cm}
	\noindent The following figure (Figure 3) delineates the diagram for the designed clock. In addition, the Verilog code for this block is provided in Appendix 3.

	\begin{figure}[H]
		\centering
		\includegraphics[height=4.5cm,width=8cm]{figures/clock}
		\caption{Clock diagram}
	\end{figure}
\vspace{-0.4cm}
	 
	\noindent \textbf{\large Stopwatch Design}
	\vspace{0.2cm}
	
	\noindent The inputs and outputs for the stopwatch are identical to the ones regarding the clock design. The differentiation to be made between the to components would be regarding both the counter that they were connected to as well as the limit for each digit. Regarding the prior, the stopwatch was connected to the Decisecond counter rather than the second counter, as it should operate on a tenth of a second basis. In addition, regarding the latter difference, the limit for all of the 4 digits is set to 9. The following figure (Figure 4) displays the diagram for the designed stopwatch. The Verilog code for this block is provided in Appendix 4.
	
	\begin{figure}[H]
		\centering
		\includegraphics[height=4cm,width=8cm]{figures/stopwatch}
		\caption{Stopwatch diagram}
	\end{figure}

	\noindent \textbf{\large Pause counter}
	\vspace{0.2cm}
	
	\noindent As mentioned, in order to make the time setting feature more convenient for the user, a 0.2 pause is to be made before each increment when the user is pressing the keys on the board. The design of the component for this pause is identical to the second and Decisecond counter, with the slight difference that this component would output a HIGH signal every 5400000 clock cycles: 0.2 seconds (check Figure 5 and Appendix 5).

	\begin{figure}[H]
		\centering
		\includegraphics[height=4.5cm,width=8cm]{figures/pause}
		\caption{Pause counter diagram}
	\end{figure}
	\noindent \textbf{\large Priority 4-to-1 Encoder}
	\vspace{0.2cm}
	
	\noindent The Altera is equipped with four keys (buttons). As decided in the design specification, these keys were dedicated to the time setting feature, where each key would cause an increment in the corresponding digit. For instance, key0 would increase the value of $Q-{0}$, key1 would increase the value of $Q_{1}$, etc. Hence, it was decided that a 4-to-1 priority encoder could be implemented. Accordingly, at any given time, only one digit would be affected, with decreasing priority in the increments from left to right. This is believed to reduce time leading/lagging errors as well as error due to occasional glitches. The block diagram for this component is provided in the below figure (Figure 6). Moreover, the Verilog code for constructing is provided in Appendix 6.
	
	\begin{figure}[H]
		\centering
		\includegraphics[height=4cm,width=8cm]{figures/encoder}
		\caption{4-to-1 priority encoder diagram}
	\end{figure}


	\noindent \textbf{\large Display selector}
	\vspace{0.2cm}
	
	\noindent As the Altera board includes merely four 7-segment displays, the simultaneous display of both the clock and the stopwatch is not practical as each of these components would require four displays respectively. Therefore, a component was to be designed to control what is shown by the four 7-segment displays. Accordingly, the display selector component would act as a 2-to-1 selector, which would receive an input from the user indicating the operation to be displayed. More specifically, when the input is set to LOW, the output of the clock would be displayed and when it is set to HIGH, the displays would show the output of the stopwatch. Furthermore, this block divides the 16-bit input to four 4-bit outputs that can be displayed on the 7-segment displays respectively. The block diagram for this component is provided in the figure below (Figure 7) and this block can be constructed by the Verilog code in Appendix 7.
	
	\vspace{-0.2cm}
	\begin{figure}[H]
		\centering
		\includegraphics[height=4.5cm,width=8cm]{figures/selector}
		\caption{Display selector diagram}
	\end{figure}

	\vspace{-0.2cm}
	\noindent \textbf{\large BCD-to-7-segment Decoder}
	\vspace{0.2cm}
	
	\noindent As implied by the name, the displays on the DE1 board are constructed of seven LED segments. Each of these segments are turned on when subject to a LOW signal. Hence, implementing a BCD-to-7segment is necessary.
	
		\vspace{-0.5cm} 
	\begin{figure}[H]
		\centering
		\includegraphics[height=4cm,width=6cm]{figures/seg}
		\caption{Segment index and position for DE1 board}
	\end{figure}
\vspace{-0.2cm}
	\noindent The Above figure illustrates the bit number corresponding to each of these segments. Accordingly, a 7-bit output for each decimal from zero to nine that displays the shape of these digits on the board was to be determined. These outputs are provided in the following table (Table 1).
	\begin{table}[H]
		\centering
		\begin{tabular}{|c | c| }
			\hline
			\textbf{Input} & \textbf{Output} \\ \hline
			0& 0000001\\\hline
			1&1001111\\\hline
			2&0010010\\\hline
			3&0000110\\\hline
			4&1001100\\\hline
		\end{tabular}
	\hspace{0.2cm}
	\begin{tabular}{|c | c| }
		\hline
		\textbf{Input} & \textbf{Output} \\ \hline
		5&0100100\\\hline
		6&0100000\\\hline
		7&0001111\\\hline
		8&0000000\\\hline
		9&0000100\\\hline
	\end{tabular}
	\caption{Translation table for decoder}
	\end{table}

	\vspace{-0.2cm}
	\noindent The code for this block is provided in Appendix 8. Moreover, the block diagram for this component is provided below (Figure 9).
	
	\begin{figure}[H]
		\centering
		\includegraphics[height=4cm,width=8cm]{figures/segment}
		\caption{BCD-to-7-segment decoder diagram}
	\end{figure}
		
	\vspace{-0.4cm}
	\noindent \textbf{\large Alarm}
	\vspace{0.2cm}
	
	\noindent In order to constantly flash a group of LED lights when the clock and/or the stopwatch reach their default state in the down counting mode, it was decided that a component is to be constructed. The input for this component would be provided by its preceding component, whether it is the clock or the stopwatch. Furthermore, this component has an n-bit output, with n corresponding to the number of LEDs on the DE1 board that have the same color. As the Altera board includes eight green LEDs and ten red LEDs, two alarm components with 8-bit and 10-bit outputs respectively were constructed. The logic for this component is significantly similar to the pause counter. That is, it would output a HIGH signal every 5400000 clock signals: 0.2 seconds. With this approach, the LED lights would flash with 0.2-second intervals. Figure 10 displays the block diagram of this component. Moreover, the Verilog code for constructing this component is provided in Appendix 9 and 10.
	
	\vspace{-0.2cm}
	\begin{figure}[H]
		\centering
		\includegraphics[height=5cm,width=8cm]{figures/alarm}
		\caption{Alarm components}
	\end{figure}
	
	\vfill
	\vspace{20cm}
	
	\noindent \textbf{\large Pin Assignments}
	\vspace{0.2cm}
	
	\noindent The following table (Table 2) displays the input pin assignments respectively.
\begin{table}[H]
	\centering
	\begin{tabular}{|c | c| }
		\hline
		\textbf{Input} & \textbf{Pin} \\ \hline
		CLK& D12\\\hline
		CLR& L21\\\hline
		ENT& L22\\\hline
		UPDOWN & M22\\\hline
		deciMode & V12\\\hline
		CLR2&U12\\\hline
		ENT2&W12\\\hline
		UPDOWN2&U11\\\hline
		W[0] & R22\\\hline
		W[1] &R21\\\hline
		W[2] &T22\\\hline
		W[3] &T21\\\hline
	\end{tabular}
	\caption{Table of Pin assignments for inputs}
\end{table}

\noindent In addtion, the following table (Table 3) includes all the pin assignments regarding the output of the design.


	\begin{table}[H]
	\centering
	\begin{tabular}{|c | c| }
		\hline
		\textbf{Output} & \textbf{Pin} \\ \hline
		D[0]& E2\\\hline
		D[1]& F1\\\hline
		D[2]& F2\\\hline
		D[3]&H1\\\hline
		D[4]& H2\\\hline
		D[5]&J1\\\hline
		D[6]&J2\\\hline
		S1[0]&D1\\\hline
		S1[1]&D2\\\hline
		S1[2]&G3\\\hline
		S1[3]&H4\\\hline
		S1[4]&H5\\\hline
	\end{tabular}
	\hspace{0.2cm}
	\begin{tabular}{|c | c| }
		\hline
		\textbf{Output} & \textbf{Pin} \\ \hline
		S1[5]&H6\\\hline
		S1[6]&E1\\\hline
		S2[0]&D3\\\hline
		S2[1]&E4\\\hline
		S2[2]&E3\\\hline
		S2[3]&C1\\\hline
		S2[4]&C2\\\hline
		S2[5]&G6\\\hline
		S2[6]&G5\\\hline
		M[0]&D4\\\hline
		M[1]&F3\\\hline
		M[2]&L8\\\hline
	\end{tabular}
	\hspace{0.2cm}
	\begin{tabular}{|c | c| }
		\hline
		\textbf{Output} & \textbf{Pin}  \\ \hline
		M[3]&J4\\\hline
		M[4]&D6\\\hline
		M[5]&D5\\\hline
		M[6]&F4\\\hline
		L[0]&R17\\\hline
		L[1]&R18\\\hline
		L[2]&R18\\\hline
		L[3]&R18\\\hline
		L[4]&R18\\\hline
		L[5]&R18\\\hline
		L[6]&R18\\\hline
		L[7]&R18\\\hline
	\end{tabular}
	\hspace{0.2cm}
	\begin{tabular}{|c | c| }
		\hline
		\textbf{Output} & \textbf{Pin}  \\ \hline
		L[8]&D5\\\hline
		L[9]&F4\\\hline
		G[0]&Y21\\\hline
		G[1]&Y22\\\hline
		G[2]&W21\\\hline
		G[3]&W22\\\hline
		G[4]&V21\\\hline
		G[5]&V22\\\hline
		G[6]&U21\\\hline
		G[7]&U22\\\hline
		&\\\hline
		&\\\hline
	\end{tabular}
	\caption{Table of Pin assignments for outputs}
\end{table}
	
	\vspace{2cm}
	\noindent \textbf{\large Circuitry}
	\vspace{0.2cm}
	
	\noindent The complete circuit, with all the wires connected and the pins assigned, is provided in the figure below (Figure 11).
	
	\begin{figure}[H]
		\centering
		\includegraphics[height=17.7cm,width=15cm]{figures/DigitalClock.jpg}
		\caption{The complete circuit}
	\end{figure}

	\vspace{2cm}
	\noindent \textbf{\Large Results and Discussion}
	\vspace{0.2cm}
	
	\noindent In this section, the simulation results, which demonstrate the functionality of each of the implemented functions, are provided respectively. As simulations allow for micro-analysis of the clock cycles, the counter components were modified to operate at a much less frequency with the same frequency ratio. More specifically, the Decisecond counter, the alarms, the pause counter,and the second counter were modified to output a HIGH signal every 1,2, 2, 10 clock cycles. With this approach, it is believed that the resulting results could be better illustrated and analyzed. In addition, the obtained results would be meticulously observed and discussed. The user interface is provided likewise. 
	
	\vspace{0.4cm}
	\noindent \textbf{\large Clock}
	\vspace{0.2cm}
	
	\noindent A functional clock is expected to operate in accordance to the design specification. In this section, the results corresponding to each function with thorough analysis is provided.
	
	\vspace{0.2cm}
	\noindent \textbf{Up Counting}
	\vspace{0.2cm}
	
	\noindent The most fundamental function of a clock is to count upwards and update the 7-segment displays accordingly. In order to simulate the results for this function, the waveforms were simulated during 10$\mu s$, 20$\mu s$, 100$\mu s$, and 420$\mu s$ duration to clearly illustrate the increments for each digit respectively. For the BCD-to-binary decoding translation, refer to Table 1. 
	
	\begin{figure}[H]
		\centering
		\includegraphics[height=6cm,width=14cm]{figures/clock1}
		\caption{Up counting for the clock ($Q_{0}$)}
	\end{figure}
	
	\noindent According to the above figure (Figure 12), the clock is able to accurately increment the right-most digit ($Q_{0}$) from 0 to 9, reset the digit to 0, and accordingly, increment $Q_{1}$ by one unit. 

	\begin{figure}[H]
		\centering
		\includegraphics[height=7.5cm,width=14cm]{figures/clock2}
		\caption{Up counting for the clock ($Q_{1}$)}
	\end{figure}

	\noindent The above figure (Figure 13) shows that the up-counting operation is conducted properly: $Q_{1}$ is accurately incremented from 0 to 5, reset to 0, and $Q_{2}$ is incremented by one unit in correspondence. Furthermore, as shown in the figure below (Figure 14), the incremental operation is successfully achieved for $Q_{2}$ and $Q_{3}$ has been updated in accordance.
	
	\begin{figure}[H]
		\centering
		\includegraphics[height=7.5cm,width=14cm]{figures/clock3}
		\caption{Up counting for the clock ($Q_{2}$)}
	\end{figure}

	\begin{figure}[H]
		\centering
		\includegraphics[height=6cm,width=14cm]{figures/clock4}
		\caption{Up counting for the clock ($Q_{3}$)}
	\end{figure}
	
	\noindent As illustrated in Figure 15, similar to the previous results, the operation successfully incremented $Q_{3}$ from 0 to 5 and reset to 0 after the limit has been reached. Therefore, it can be concluded that the up-counting operation for the clock is functional. 

	
	\vspace{0.3cm}
	\noindent \textbf{Time setting}
	\vspace{0.2cm}
	
	\noindent The second significant function of a functional clock is providing the ability to set the time. The waveforms for this function were simulated for duration of 400$ns$. The time setting feature was tested for all four digits and the results are provided below respectively.
	
	\begin{figure}[H]
		\centering
		\includegraphics[height=7cm,width=14cm]{figures/set1}
		\caption{Time setting for the clock ($Q_{0}$)}
	\end{figure}

	\noindent The above figure (Figure 16) indicates that by holding down KEY0 (setting W[0] to LOW), $Q_{0}$ is incremented from 0 to 9 by units of one and reset to 0 after reaching 9, which corresponds to the design specification. In addition, the below figure (Figure 17) illustrates that the incremental operation is accurately conducted for $Q_{1}$ likewise; that is, the digits are incremented from 0 to 5 and reset to 0 when the limit is reached.

	\begin{figure}[H]
		\centering
		\includegraphics[height=6.5cm,width=14cm]{figures/set2}
		\caption{Time setting for the clock ($Q_{1}$)}
	\end{figure}

	\noindent According to the below figure (Figure 18), $Q_{2}$ is incremented by units of one from 0 to 9 and reset to 0 when the digit has reached 9. 

	\begin{figure}[H]
		\centering
		\includegraphics[height=7cm,width=14cm]{figures/set3}
		\caption{Time setting for the clock ($Q_{2}$)}
	\end{figure}

	\begin{figure}[H]
		\centering
		\includegraphics[height=6cm,width=14cm]{figures/set4}
		\caption{Time setting for the clock ($Q_{3}$)}
	\end{figure}
	
	\noindent The above figure (Figure 19) illustrates that the incremental operation is performed accurately and hence, as the operation has been functional for all the displays ($Q_{0-3}$), it can be concluded that the time setting feature is successfully functioning.
	
	\vspace{0.2cm}
	\noindent \textbf{Down Counting}
	\vspace{0.2cm}
	
	\noindent Designed as an additional feature, the down-counting operation is performed similarly to the up-counting operating, with decrements in the digits rather then increments. In order to simulate results for the down-counting process and provide a clear illustration of the performance, the time-setting feature must initially be operated. Therefore, for each digit's analysis, an increment is simulated in prior and accordingly, the down-counting process for each digit is observed respectively.
	
	\begin{figure}[H]
		\centering
		\includegraphics[height=6cm,width=14cm]{figures/clock5}
		\caption{Down counting for the clock ($Q_{0}$)}
	\end{figure}

	\noindent According to the above figure (Figure 20), $Q_{0}$ is initially incremented to 2 and by after the UPDOWN input is set to HIGH, the value of $Q_{0}$ is decremented by units of one from 2 to 0.

	\begin{figure}[H]
		\centering
		\includegraphics[height=6cm,width=14cm]{figures/clock6}
		\caption{Down counting for the clock ($Q_{1}$)}
	\end{figure}

	\noindent Evidently, Figure 21 indicates the same results for $Q_{1}$, as the decrements are performed accurately. Additionally, it can be observed that after $Q_{1}$ is decremented, the value of $Q_{0}$ is set to 9 and the down-counting process effectively proceeds. The same observation can be made from the below figure (Figure 22), where each decrement in $Q_{2}$ causes $Q_{1}$ to be set to 5. 
	
	\begin{figure}[H]
		\centering
		\includegraphics[height=6cm,width=14cm]{figures/clock7}
		\caption{Down counting for the clock ($Q_{2}$)}
	\end{figure}

	\noindent As illustrated in the below figure (Figure 23), the down-counting operation for $Q_{3}$ is performed correctly. From this Figure, it can be observed that $Q_{3}$ is properly decremented until it reaches 0 and correspondingly, sets the value of $Q_{2}$ to 9. Hence, based on the obtained results, it is believed that the down-counting process is fully functional.
	
	\begin{figure}[H]
		\centering
		\includegraphics[height=6cm,width=14cm]{figures/clock8}
		\caption{Down counting for the clock ($Q_{3}$)}
	\end{figure}
		
	
	\noindent \textbf{Reset}
	\vspace{0.2cm}
	
	\noindent The results for the reset feature are provided in the below figure (Figure 24). As illustrated in the figure, all the digits are set to 0 when the CLR input is set to HIGH. Hence, this feature is properly functioning.
	
	
	\begin{figure}[H]
		\centering
		\includegraphics[height=6cm,width=14cm]{figures/reset1}
		\caption{Reset function for clock}
	\end{figure}
		
	\noindent \textbf{Alarm}
	\vspace{0.2cm}
	
	\noindent As displayed in Figure 25, when the clock reaches 00:00 in the down-counting mode, output G is continuously set to HIGH (green LEDs are turned on) and reset to LOW (green LEDs are turned off). This is an accurate demonstration of the flashing effect, as requested by the design specifications.
		
	\begin{figure}[H]
		\centering
		\includegraphics[height=5cm,width=14cm]{figures/alarm1}
		\caption{Alarm function for clock}
	\end{figure}
	
	\vspace{0.2cm}
	\noindent \textbf{\large Stopwatch}
	\vspace{0.2cm}
	
	\noindent A functional stopwatch is expected to be able to perform the same functions as the clock. In this section, the results corresponding to each function with thorough analysis is provided. For all of the following simulations, deciMode is set to HIGH as functions of the stopwatch are to be analyzed. 
	
	\vspace{0.2cm}
	\noindent \textbf{Up Counting}
	\vspace{0.2cm}
	
	\noindent Similar to the clock, the most fundamental function of a stopwatch is also to count upwards and update the 7-segment displays accordingly. In order to simulate the results for this function, the waveforms were simulated during 1$\mu s$, 2$\mu s$, 20$\mu s$, and 200$\mu s$ duration to illustrate the increments for each digit respectively. For the BCD-to-binary decoding translation, refer to Table 1. 
	
	\begin{figure}[H]
		\centering
		\includegraphics[height=5cm,width=14cm]{figures/dclock1}
		\caption{Up counting for stopwatch ($Q_{0}$)}
	\end{figure}
	
	\noindent As displayed in the above figure (Figure 26), the stopwatch is able to accurately increment the right-most digit ($Q_{0}$) from 0 to 9, reset the digit to 0, and accordingly, increment $Q_{1}$ by one unit. 
	
	\begin{figure}[H]
		\centering
		\includegraphics[height=6.5cm,width=14cm]{figures/dclock2}
		\caption{Up counting for stopwatch ($Q_{1}$)}
	\end{figure}
	
	\noindent The above figure (Figure 27) shows that the up-counting operation is properly functioning: $Q_{1}$ is accurately incremented from 0 to 9, reset to 0, and $Q_{2}$ is incremented by one unit in correspondence. In addition, as displayed in the figure below (Figure 28), the incremental operation is successfully achieved for $Q_{2}$ and $Q_{3}$ has been updated in correspondence .
	
	\begin{figure}[H]
		\centering
		\includegraphics[height=6.5cm,width=14cm]{figures/dclock3}
		\caption{Up counting for stopwatch ($Q_{2}$)}
	\end{figure}
	
	\begin{figure}[H]
		\centering
		\includegraphics[height=6cm,width=14cm]{figures/dclock4}
		\caption{Up counting for stopwatch ($Q_{3}$)}
	\end{figure}
	
	\noindent By observing Figure 29 similar results to the previous simulations is derived:  $Q_{3}$ is successfully incremented from 0 to 9 and reset to 0 after the limit has been reached. Therefore, it can be concluded that the up-counting operation for the stopwatch is functional. 
	
	
	\vspace{0.3cm}
	\noindent \textbf{Time setting}
	\vspace{0.2cm}
	
	\noindent The stopwatch should also be able to perform the time setting function. Similarly, the waveforms for this function were simulated for 400$ns$, the time setting feature was tested for all four digits and the results are provided below respectively.
	
	\begin{figure}[H]
		\centering
		\includegraphics[height=6cm,width=14cm]{figures/set5}
		\caption{Time setting for stopwatch ($Q_{0}$)}
	\end{figure}
	
	\noindent Figure 30 clearly displays that by holding down KEY0 (setting W[0] to LOW), $Q_{0}$ is incremented from 0 to 9 by units of one and reset to 0 after reaching 9. Moreover, the below figure (Figure 31) indicates that the incremental operation is accurately performed for $Q_{1}$ as well: the digits are incremented from 0 to 9 and reset to 0 when the limit is reached.
	
	\begin{figure}[H]
		\centering
		\includegraphics[height=6.5cm,width=14cm]{figures/set6}
		\caption{Time setting for stopwatch ($Q_{1}$)}
	\end{figure}
	
	\noindent According to the below figure (Figure 32), $Q_{2}$ is incremented by units of one from 0 to 9 and reset to 0 when the digit has reached 9. 
	
	\begin{figure}[H]
		\centering
		\includegraphics[height=7cm,width=14cm]{figures/set7}
		\caption{Time setting for stopwatch ($Q_{2}$)}
	\end{figure}
	
	\begin{figure}[H]
		\centering
		\includegraphics[height=6cm,width=14cm]{figures/set8}
		\caption{Time setting for stopwatch ($Q_{3}$)}
	\end{figure}
	
	\noindent The above figure (Figure 33) displays that the incremental operation is performed accurately and hence, as the operation has been functional for all the displays ($Q_{0-3}$). Therefore, the time setting feature for stopwatch is functional.
	
	\vspace{0.2cm}
	\noindent \textbf{Down Counting}
	\vspace{0.2cm}
	
	\noindent The down-counting operation in the stopwatch is performed similarly to the up-counting operation, with decrements in the digits rather then increments. In order to simulate the results for the down-counting process and provide a clear illustration of the performance, the time-setting feature must initially be operated. Therefore, for each digit's analysis, an increment is simulated in prior and correspondingly, the down-counting process for each digit is observed respectively.
	
	\begin{figure}[H]
		\centering
		\includegraphics[height=6cm,width=14cm]{figures/dclock5}
		\caption{Down counting for stopwatch ($Q_{0}$)}
	\end{figure}
	
	\noindent According to the above figure (Figure 34), $Q_{0}$ is initially incremented to 5 and by after the UPDOWN input is set to HIGH, the value of $Q_{0}$ is decremented by units of one from 5 to 0.
	
	\begin{figure}[H]
		\centering
		\includegraphics[height=6cm,width=14cm]{figures/dclock6}
		\caption{Down counting for stopwatch ($Q_{1}$)}
	\end{figure}
	
	\noindent Figure 35 indicates the same results for $Q_{1}$, as the decrements are performed accurately. Moreover, it can be derived that after $Q_{1}$ is decremented, the value of $Q_{0}$ is set to 9 and the down-counting process effectively proceeds. The same observation can be made from the below figure (Figure 36), where each decrement in $Q_{2}$ causes $Q_{1}$ to be set to 9. 
	
	\begin{figure}[H]
		\centering
		\includegraphics[height=6cm,width=14cm]{figures/dclock7}
		\caption{Down counting for stopwatch ($Q_{2}$)}
	\end{figure}
	
	\noindent As shown in the below figure (Figure 37), the down-counting operation for $Q_{3}$ is performed correctly. From this Figure, it can be observed that $Q_{3}$ is properly decremented until it reaches 0 and correspondingly, sets the value of $Q_{2}$ to 9. Hence, based on the obtained results, it is believed that the down-counting process is functional.
	
	\begin{figure}[H]
		\centering
		\includegraphics[height=6cm,width=14cm]{figures/dclock8}
		\caption{Down counting for stopwatch ($Q_{3}$)}
	\end{figure}
	
	\noindent 
	
	\noindent \textbf{Reset}
	\vspace{0.2cm}
	
	\noindent The results for the reset feature are provided in the below figure (Figure 38). As displayed in the figure, all the digits are set to 0 when the CLR input is set to HIGH. Hence, this feature is properly functioning.
	
	
	\begin{figure}[H]
		\centering
		\includegraphics[height=6cm,width=14cm]{figures/reset2}
		\caption{Reset function for stopwatch}
	\end{figure}
	
	\noindent \textbf{Alarm}
	\vspace{0.2cm}
	
	\noindent As shown in Figure 39, when the stopwatch reaches 000:0 in the down-counting mode, output L is continuously set to HIGH (red LEDs are turned on) and reset to LOW (red LEDs are turned off), which corresponds to the desired flashing effect.
	
	\begin{figure}[H]
		\centering
		\includegraphics[height=6cm,width=14cm]{figures/alarm2}
		\caption{Alarm function for stopwatch}
	\end{figure}
	
	
	\vspace{0.2cm}
	\noindent \textbf{\large Simultaneous operation}
	\vspace{0.2cm}
	
	\noindent As specified in the design objectives, simultaneous operation of both the clock and the stopwatch was required. The following figure (Figure 40) illustrates the results for this operation for simulation with duration of $20\mu s$. As displayed in the figure, the displayed outputs correspond to the clock when deciMode is set to LOW, and they correspond to the stopwatch  when deciMode is set to HIGH. Hence, the implementation of this feature is successfully achieved.
	
	\begin{figure}[H]
		\centering
		\includegraphics[height=6cm,width=14cm]{figures/multi}
		\caption{Simultaneous operation of clock and stopwatch}
	\end{figure}
	
	\vspace{0.2cm}
	\noindent \textbf{\large User Interface}
	\vspace{0.2cm}
	
	\noindent The complete user manual to access all of the functions of the design is provided in the table below (Tables 4 and 5). 
	
	\begin{table}[H]
		\centering
		\begin{tabular}{|c | c| }
			\hline
			\textbf{Switch} & \textbf{Function} \\ \hline
			SW0& Start the clock (switch down to pause)\\\hline
			SW1& Reset the clock\\\hline
			SW2& Change the counting direction for clock (Up for low counting)\\\hline
			SW3& Change the display (Down for MM:SS and up for SSS:Ds)\\\hline
			SW4& Start the stopwatch (switch down to pause)\\\hline
			SW5& Reset the stopwatch\\\hline
			SW6& Change the counting direction for stopwatch (Up for low counting)\\\hline
		\end{tabular}
		\caption{User interface manual for switches}
	\end{table}

	\vspace{-0.4cm}
	\begin{table}[H]
		\centering
		\begin{tabular}{|c | c| }
			\hline
			\textbf{Key} & \textbf{Function} \\ \hline
			KEY0& increment the value of Q0 by 1\\\hline
			KEY1& increment the value of Q1 by 1\\\hline
			KEY2& increment the value of Q2 by 1\\\hline
			KEY3& increment the value of Q3 by 1\\\hline
		\end{tabular}
		\caption{User interface manual for keys}
	\end{table}
	
	\vspace{-0.2cm}
	\noindent Note that in order for increments to be functioning, the counter must be paused in prior. The keys will not have any affect on an operating clock/stopwatch.
	
	\vspace{0.5cm}
	\noindent \textbf{\Large Conclusion}
	\vspace{0.2cm}
	
	
	\noindent In this assignment, a digital clock circuit was designed and implemented. The clock had two main components: normal clock (MM:SS) and stopwatch (SSS:D). As regards to the functionality, all of the required functions as well as the additional functions were designed and implemented with success. It is believed that after successful completion of this assignment, valuable experience regarding the Quartus II software, the DE1 Altera board, and Verilog was obtained. In addition, it is believed that deeper understanding of the lecture materials, which were taught in the EEE339 module, was achieved. Conclusively, this assignment was considered a valuable learning opportunity as well as a successful implementation of a functional digital clock.
	
	\vspace{0.2cm}
	\noindent \textbf{\Large References}
	\vspace{0.2cm}
	
	\noindent [1] \textit{EEE205 Lab Altera Experiment}, 4 ed., Xi’an Jiaotong Liverpool University, Department of Electrical and Electronic Engineering, Suzhou, 2008
	
	\vspace{0.2cm}
	\vfill
\vspace{10cm}
	
	\noindent \textbf{\Large Appendices}
	\vspace{0.2cm}
	
	\begin{lstlisting}[style={verilog-style},caption={Verilog code for Decisecond counter},captionpos=b]
module deciCount (clk, Q);
input clk;
output reg Q;
reg [25:0] count = 0; // 26 bit output to store 2700000

always @(posedge clk)
begin
Q <= 0;
if (count == 26'd2700000) // 0.1 second is reached
begin
count <= 0;
Q <= 1;
end
else
count <= count+1;
end
endmodule
	\end{lstlisting}
	
	\begin{lstlisting}[style={verilog-style},caption={Verilog code for second counter},captionpos=b]
module secondCount (clk, Q);
input clk;
output reg Q;
reg [25:0] count = 0; // 26 bit output to store 27000000

always @(posedge clk)
begin
Q <= 0;
if (count == 26'd27000000) // 1 second is reached
begin
count <= 0;
Q <= 1;
end
else
count <= count+1;
end
endmodule
	\end{lstlisting}
	

	
	\begin{lstlisting}[style={verilog-style},caption={Verilog code for clock},captionpos=b]
module normalClock (Clk, Pause, Enc, Ent, Clr, UpDown, 
deciMode, timeSetter, Q, Rco);
input Clk, Pause, Enc, Ent, Clr, UpDown, deciMode;
input [3:0] timeSetter;
reg[3:0] Q1; // output for Q0 display
reg[3:0] Q2; // output for Q1 display
reg[3:0] Q3; // output for Q2 display
reg[3:0] Q4; // output for Q3 display
output reg [15:0] Q; // The cumulative output
output reg Rco =0;

always @(posedge Clk, posedge Clr)
begin
if (Clr)
begin
Q1 <= 4'b0000;
Q2 <= 4'b0000;
Q3 <= 4'b0000;
Q4 <= 4'b0000;
Rco <= 0;
end
else if (Enc && Ent)
begin
Rco <= 0;
if (!UpDown) //Incremental
begin 
if (Q1==4'b1001) //If the first digit is 9 (LSB)
begin
Q1 <= 4'b0000;
if (Q2 == 4'b0101) //If the second digit is 5
begin
Q2 <= 4'b0000;
begin 
if (Q3==4'b1001) //If the third digit is 9
begin
Q3 <= 4'b0000;
if (Q4 == 4'b0101) //If the fourth digit is 5
Q4 <= 4'b0000;
else	
Q4 <= Q4+1;
end
else
Q3 <= Q3+1;
end
end
else	
Q2 <= Q2+1;
end
else
Q1 <= Q1+1;
end
else //Decremental
begin 
if (Q1==4'b0000)
begin
if (Q2==4'b0000 && Q3==4'b0000 && Q4==4'b0000)
Rco <= 1;
else
begin
Q1 <= 4'b1001;
if (Q2 == 4'b0000)
begin
Q2 <= 4'b0101;
begin 
if (Q3==4'b0000)
begin
Q3 <= 4'b1001;
if (Q4 == 4'b0000)
Q4 <= 4'b0101;
else	
Q4 <= Q4-1;
end
else
Q3 <= Q3-1;
end
end
else	
Q2 <= Q2-1;
end
end

else
Q1 <= Q1-1;
end
end
else if (!Ent &&Pause && !deciMode) // for time setting
begin
case (timeSetter)
4'b0001: begin 
if (Q1 == 4'b1001)
Q1 <= 4'b0000;
else
Q1 <= Q1+1;
end
4'b0010: begin
if (Q2 == 4'b0101)
Q2 <= 4'b0000;
else	
Q2 <= Q2+1;
end
4'b0100: begin
if (Q3 == 4'b1001)
Q3 <= 4'b0000;
else	
Q3 <= Q3+1;
end
4'b1000: begin
if (Q4 == 4'b0101)
Q4 <= 4'b0000;
else	
Q4 <= Q4+1;
end
endcase
end
Q <= {Q4, Q3,Q2,Q1};
end
endmodule
	
	\end{lstlisting}
	
		\vspace{-0.4cm}
	
	\begin{lstlisting}[style={verilog-style},caption={Verilog code for stopwatch},captionpos=b]
module stopwatch (Clk, Pause, Enc, Ent, Clr, UpDown, deciMode
,timeSetter, Q, Rco);
input Clk, Pause, Enc, Ent, Clr, UpDown, deciMode;
input [3:0] timeSetter;
reg[3:0] Q1; // output for Q0 display
reg[3:0] Q2; // output for Q11 display
reg[3:0] Q3; // output for Q2 display
reg[3:0] Q4; // output for Q3 display
output reg [15:0] Q; // The cumulative output
output reg Rco =0;

always @(posedge Clk, posedge Clr)
begin
if (Clr)
begin
Q1 <= 4'b0000;
Q2 <= 4'b0000;
Q3 <= 4'b0000;
Q4 <= 4'b0000;
Rco <= 0;
end
else if (Enc && Ent)
begin
Rco <= 0;
if (!UpDown) //Incremental
begin 
if (Q1==4'b1001) //If the first digit is 9 (LSB)
begin
Q1 <= 4'b0000; //Set it to zero
if (Q2 == 4'b1001) //If the second digit is 9
begin
Q2 <= 4'b0000;
begin 
if (Q3==4'b1001) //If the third digit is 9
begin
Q3 <= 4'b0000;
if (Q4 == 4'b1001) //If the fourth digit is 9
Q4 <= 4'b0000;
else	
Q4 <= Q4+1;
end
else
Q3 <= Q3+1;
end
end
else	
Q2 <= Q2+1;
end
else
Q1 <= Q1+1;
end
else //Decremental
begin 
if (Q1==4'b0000)
begin
if (Q2==4'b0000 && Q3==4'b0000 && Q4==4'b0000)
Rco <= 1;
else
begin
Q1 <= 4'b1001;
if (Q2 == 4'b1001)
begin
Q2 <= 4'b1001;
begin 
if (Q3==4'b0000)
begin
Q3 <= 4'b1001;
if (Q4 == 4'b1001)
Q4 <= 4'b1001;
else	
Q4 <= Q4-1;
end
else
Q3 <= Q3-1;
end
end
else	
Q2 <= Q2-1;
end
end

else
Q1 <= Q1-1;
end
end
else if (!Ent &&Pause &&deciMode) // Time setting
begin
case (timeSetter)
4'b0001: begin 
if (Q1 == 4'b1001)
Q1 <= 4'b0000;
else
Q1 <= Q1+1;
end
4'b0010: begin
if (Q2 == 4'b1001)
Q2 <= 4'b0000;
else	
Q2 <= Q2+1;
end
4'b0100: begin
if (Q3 == 4'b1001)
Q3 <= 4'b0000;
else	
Q3 <= Q3+1;
end
4'b1000: begin
if (Q4 == 4'b1001)
Q4 <= 4'b0000;
else	
Q4 <= Q4+1;
end
endcase
end
Q <= {Q4, Q3,Q2,Q1};
end
endmodule
	
	\end{lstlisting}
	
	\vspace{-0.4cm}
	
		\begin{lstlisting}[style={verilog-style},caption={Verilog code for pause counter},captionpos=b]
module pauseCount (clk, Q);
input clk;
output reg Q;
reg [25:0] count = 0; //26 output to store 5400000

always @(posedge clk)
begin
Q <= 0;
if (count == 26'd5400000) // 0.2 seconds reached
begin
count <= 0;
Q <= 1;
end
else
count <= count+1;
end
endmodule
	\end{lstlisting}
	
	
		\begin{lstlisting}[style={verilog-style},caption={Verilog code for 4-to-1 priority encoder},captionpos=b]
module enc4to1 (W, Y);
input [3:0] W;
output reg [3:0] Y;

always @(W)
begin 
casex(W)
4'b0xxx: Y=4'b1000; // if key3 is pressed
4'b10xx: Y=4'b0100; // if key2 is pressed
4'b110x: Y=4'b0010; // if key1 is pressed
4'b1110: Y=4'b0001; // if key0 is pressed
default: Y=4'b0000; // no key is pressed
endcase
end
endmodule
	\end{lstlisting}
	
	\begin{lstlisting}[style={verilog-style},caption={Verilog code for display selector},captionpos=b]
module displaySelector (deciMode,R1, R2, Q1, Q2, Q3, Q4);
input deciMode;
input [15:0] R1; // 16 bit input from the clock
input [15:0] R2; // 16 bit input from the stopwatch
output reg [3:0] Q1; // output for Q0 display
output reg [3:0] Q2; // output for Q1 display
output reg [3:0] Q3; // output for Q2 display
output reg [3:0] Q4; // output for Q3 display

always @(deciMode)
if (!deciMode) // clock is displayed
{Q4, Q3, Q2, Q1}= R1;
else // stopwatch is displayed
{Q4, Q3, Q2, Q1}= R2;
endmodule
\end{lstlisting}	

	\begin{lstlisting}[style={verilog-style},caption={Verilog code for 7-segment decoder},captionpos=b]
module seg7Decoder(bcd, leds);
input[3:0] bcd;
output reg[7:1] leds;

always @(bcd)
case(bcd) 
4'b0000: leds= 7'b0000001; //Displaying 0
4'b0001: leds= 7'b1001111; //Displaying 1
4'b0010: leds= 7'b0010010; //Displaying 2
4'b0011: leds= 7'b0000110; //Displaying 3
4'b0100: leds= 7'b1001100; //Displaying 4
4'b0101: leds= 7'b0100100; //Displaying 5
4'b0110: leds= 7'b0100000; //Displaying 6
4'b0111: leds= 7'b0001111; //isplaying 7
4'b1000: leds= 7'b0000000; //Displaying 8
4'b1001: leds= 7'b0000100; //Displaying 9
default: leds= 7'bx; //Error! input is outside [0,9] range
endcase
endmodule
	\end{lstlisting}

	\begin{lstlisting}[style={verilog-style},caption={Verilog code for display selector},captionpos=b]
module alarmWatch (Clk, Ent, leds);
input Clk, Ent;
output reg [7:0] leds =0; // for the 8 green LEDs
reg [22:0] count = 0;

always @(posedge Clk)
begin
if (Ent)
begin
if (count == 0)
begin
leds <= ~leds;
end
count <= count +1;
end
else
begin
leds <= 0;
count <= 1;
end
end
endmodule
\end{lstlisting}
	
	\begin{lstlisting}[style={verilog-style},caption={Verilog code for alarm},captionpos=b ,label=DescriptiveLabel]
module alarmClock (Clk, Ent, leds);
input Clk, Ent;
output reg [9:0] leds =0; // for the 10 red LEDs
reg [22:0] count = 0;

always @(posedge Clk)
begin
if (Ent)
begin
if (count == 0)
begin
leds <= ~leds;
end
count <= count +1;
end
else
begin
leds <= 0;
count <= 1;
end
end
endmodule
	\end{lstlisting}
	

\end{document}